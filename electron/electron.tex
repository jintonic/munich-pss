\begin{filecontents}{leer.eps}
%!PS-Adobe-2.0 EPSF-2.0
%%CreationDate: Mon Jul 13 16:51:17 1992
%%DocumentFonts: (atend)
%%Pages: 0 1
%%BoundingBox: 72 31 601 342
%%EndComments

gsave
72 31 moveto
72 342 lineto
601 342 lineto
601 31 lineto
72 31 lineto
showpage
grestore
%%Trailer
%%DocumentFonts: Helvetica
\end{filecontents}
%
\documentclass[epj,draft,nopacs]{svjour}
\usepackage{graphicx}
%
\begin{document}
%
\title{Validation of pulse shape simulation for segmented germanium detectors, I}
\author{I.~Abt \and A.~Caldwell \and D.~Lenz \and J.~Liu \and B.~Majorovits}
%
%\offprints{}          % Insert a name or remove this line
%
\institute{Max-Planck-Institut f\"ur Physik, M\"unchen, Germany}
%
\date{Received: date / Revised version: date}
% The correct dates will be entered by Springer
%
\abstract{
introduce the simulation methods. summarize the comparison between data and simulation for electron drift.
%
\PACS{
  {23.40.-s}{beta decay; double beta decay; electron and muon capture} \and 
  {14.60.Pq}{Neutrino mass and mixing} \and 
  {29.40.Wk}{Solid-state detectors} \and
  {61.72.S-}{Impurities in crystals} \and
  {61.72.uf}{Ge and Si}
} % end of PACS codes
} %end of abstract
%
\maketitle
%
\section{Introduction}
\label{intro}
Your text comes here. Separate text sections with
\section{Section title}
\label{sec:1}
and \cite{RefJ}
\subsection{Subsection title}
\label{sec:2}
as required. Don't forget to give each section
and subsection a unique label (see Sect.~\ref{sec:1}).
%
% For one-column wide figures use
\begin{figure}
\includegraphics[width=0.75\textwidth]{leer.eps}
\caption{Please write your figure caption here}
\label{fig:1}       % Give a unique label
\end{figure}
%
% For two-column wide figures use
\begin{figure*}
% Use the relevant command for your figure-insertion program
% to insert the figure file. See example above.
% If not, use
\vspace*{5cm}       % Give the correct figure height in cm
\caption{Please write your figure caption here}
\label{fig:2}       % Give a unique label
\end{figure*}
%
% For tables use
\begin{table}
\caption{Please write your table caption here}
\label{tab:1}       % Give a unique label
% For LaTeX tables use
\begin{tabular}{lll}
\hline\noalign{\smallskip}
first & second & third  \\
\noalign{\smallskip}\hline\noalign{\smallskip}
number & number & number \\
number & number & number \\
\noalign{\smallskip}\hline
\end{tabular}
% Or use
\vspace*{5cm}  % with the correct table height
\end{table}
%
% BibTeX users please use
% \bibliographystyle{}
% \bibliography{}
%
% Non-BibTeX users please use
\begin{thebibliography}{}
%
% and use \bibitem to create references.
%
\bibitem{RefJ}
% Format for Journal Reference
Author, Journal \textbf{Volume}, (year) page numbers.
% Format for books
\bibitem{RefB}
Author, \textit{Book title} (Publisher, place year) page numbers
% etc
\end{thebibliography}


\end{document}
